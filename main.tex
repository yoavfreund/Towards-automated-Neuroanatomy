\documentclass{article}
\usepackage[utf8]{inputenc}

\title{Towards automated Neuroanatomy}
\author{Kui Qian, Elizabeth Friedman, David Kleinfeld, Yoav Freund}
\date{January 2022}

\begin{document}

\maketitle

\begin{abstract}
A fundamental goal of neuroanatomy is the identification of brain structures. 
Manual identification of structures is based on the spatial distribution of cell shape, size, orientation and density. 
With new technology it is possible to image entire brains at high resolution. 
However, manual identification of structures in these massive datasets is prohibitively time consuming. 
We present a machine learning method for automatic detection of brain structures. 
Our approach is to use, as a basic unit, the images of individual cells. 
We translate each cell image into a feature vector that includes aspect ratio, orientation and area, as well as additional features derived using a graph Laplacian. The algorithm uses the statistical distribution of these features vectors to identify brain structures.
\end{abstract}

\section{Introduction}

\section{Parametrization of Cell Shapes}

\section{Supervised detection of landmarks}

\section{Computer generated suggestions of new landmarks}
\end{document}
